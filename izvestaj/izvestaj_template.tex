\documentclass[a4paper, 12pt]{article}
\usepackage[utf8]{inputenc}
\usepackage[T1]{fontenc}
\usepackage[serbian]{babel}
\usepackage{amsmath}
\usepackage[margin = 2cm]{geometry}
\usepackage{graphicx}
\usepackage{nopageno}
\usepackage{blindtext}
\usepackage{tabularx}
\usepackage{multirow}
\usepackage{subcaption}



\newcolumntype{R}{>{\raggedleft\arraybackslash}X}
\newcolumntype{C}{>{\centering\arraybackslash}X}

\graphicspath{ {./img/} }

\pagestyle{empty}
\newcommand{\btmline}{
\vfill
\rule{0.9\textwidth}{0.4mm}
\begin{center}
13E043RE - Računarska elektronika
\end{center}}

\title{
\Large{Univerzitet u Beogradu - Elektrotehnički fakultet}\\
\vspace{0.5cm}
\large{Katedra za elektroniku}\\
\vspace{1.2cm}
\begin{figure}[h!]
\centering
\includegraphics[scale=1.2]{logo}
\end{figure}
\vspace{3cm}
\Huge{\textbf{\textit{Huffman Coding}}} \\
\vspace{0.5cm}
\Large{\textbf{Implementacija u MASM asemblerskom jeziku}}\\
\vspace{2cm}
\Large{13E043RE - Računarska elektronika}\\
\vfill
}
\author{Radomir Vranjevac 2017/0103\\Dušan Ilić 2017/0070}
\date{}


\renewcommand{\arraystretch}{1}
%==========================================================================
\begin{document}
%==========================TITLE===========================================
\maketitle
\newpage
%==========================================================================

\section*{Zadatak}

Potrebno je napisati program u \textbf{MASM} jeziku koji vrši \textit{Huffman}-ovo kodovanje simbola. Ovo
kodovanje zasnovano je na verovatnoći pojavljivanja simbola gde se simboli sa najvećom verovatnoćom
pojavljivanja koduju sa kodnom rečju najmanje dužine.

Nakon pokretanja programa od korisnika se traži da unese naziv tekstualne datoteke nad kojom želi
da primeni mehanizam \textit{Huffman}-ovog kodovanja. Kao rezultat kodovanja na konzoli se ispisuje
učestanost pojavljivanja pojedinih simbola kao i kodne reči koje su dodeljene pojedinom simbolu.

\btmline\newpage
%==========================================================================

\section*{Uvod}

\textit{Huffman}-ovo kodovanje predstavlja algoritam za kodovanje simbola, gde dužina koda svakog simbola zavisi
od učestanosti ponavljanja datog simbola u određenom tekstu, odnosno tekstualnoj datoteci. Cilj je da se kodovanjem
smanji potreban memorijski prostor za čuvanje sadržaja nekog fajla.

\textit{Huffman}-ovo kodovanje spada u \textit{prefix} kodove, odnosno zadovoljava pravilo da nijedan kod nije prefiks nekog drugog koda
(kodovi u kojima je svaki simbol predstavljen istim brojem bita automatski zadovoljavaju pravilo prefiksa).
Ovo predstavlja dobru osobinu jer znatno olakšava dekodovanje.

U daljem tekstu težićemo da detaljno pojasnimo postupak kodovanja.

\subsection*{Struktura projekta}

Prilikom izrade projekta, projekat je podeljen na logičke celine. Ceo projekat se sastoji iz sledećih delova i fajlova:

\begin{enumerate}
\item \textbf{Konfiguracija} - sadrži fajl u kojem se nalaze definicije konstanti i parametara korišćenih u projektu, poput maksimalne dužine niza itd.
	\begin{itemize}
	\item \verb|configuration.inc|
	\end{itemize}
	\item \textbf{Složene strukture} - sadrži fajl u kojem se nalaze definicije složenih struktura podataka koje su korišćene prilikom izrade projekta.
	\begin{itemize}
	\item \verb|structures.inc|
	\end{itemize}
\item \textbf{Rad sa tekstualnom datotekom} - sadrži fajlove za deklaracije i definicije procedura korišćenih za obradu sadržaja tekstualne datoteke.
	\begin{itemize}
	\item \verb|file_process.inc|
	\item \verb|file_process.asm|
	\end{itemize}
\item \textbf{Formiranje \textit{Huffman}-ovog stabla i rad sa njim} - sadrži deklaracije i definicije procedura korišćenih za rad sa \textit{Huffman}-ovim stablom.
	\begin{itemize}
	\item \verb|huffman_tree.inc|
	\item \verb|huffman_tree.asm|
	\end{itemize}
\item \textbf{Glavni program} - sadrži glavni program koji koristi sve pomoćne procedure definisane u ostalim fajlovima.
	\begin{itemize}
	\item \verb|main.asm|
	\end{itemize}
\end{enumerate}

\btmline\newpage
%==========================================================================

\subsubsection*{\textit{Include} fajlovi}
Svaki \verb|.inc| fajl zaštićen je od višestrukog uključivanja u projekat definisanjem makroa, čije se postojanje proverava prilikom uključivanja u projekat.
Primer koda koji ovo obezbeđuje prikazan je u nastavku.

\begin{verbatim}
ifndef _FILENAME_INC_
_FILENAME_INC_ = 0

	; content goes here
	
endif
\end{verbatim}

\subsubsection*{Strukture podataka}
Grafički prikaz struktura koje su korišćene prikazan je na slici \ref{struct}. 

\begin{figure}[h!]
\centering
\includegraphics[width=.5\textwidth]{structures}
\caption{Strukture podataka}
\label{struct}
\end{figure}

Polje \verb|symbol| strukture \verb|Data_S| sadrži karakter, a polje \verb|number| sadrži njegov broj ponavljanja.
Struktura \verb|Node_S| predstavlja čvor u \textit{Huffman}-ovom stablu . Polja \verb|left| i \verb|right| koriste se 
za određivanje deteta datog čvora. Detaljnije o ovome u nastavku.

\btmline\newpage
%==========================================================================

\section*{Rad sa datotekom}



\btmline\newpage
%==========================================================================
\section*{Formiranje stabla}
\btmline\newpage
%==========================================================================
\section*{Dobijanje koda na osnovu stabla}


\btmline\newpage
%==========================================================================
\end{document}